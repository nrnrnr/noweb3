\documentclass[11pt,a4paper]{article}% ===> this file was generated automatically by noweave --- better not edit it
\usepackage{noweb}
\usepackage{times}
\usepackage{fancyhdr}
\usepackage{graphics}
\usepackage{latexsym}
\pagestyle{noweb}
\noweboptions{}
\begin{document}
\pagestyle{headings}

\nwfilename{spawnvpe.tmp}\nwbegindocs{1}\nwdocspar
\title{\_spawnvpe, \_spwanvp}
  \author{Kai Uwe Karl Baumbach\\
  Email: baumbach@kabece.com}
  
\maketitle

\tableofcontents

% spawnvpe.nw
% June 2000 by Kai Uwe Karl Baumbach, baumbach@kabece.com
\nwenddocs{}\nwbegindocs{2}\nwdocspar
\section {Introduction}

Why should one want a {\Tt{}spawn()} function for Unix when a 
{\Tt{}fork()}/{\Tt{}exec()} combination is much more flexible and
powerful? In a large amount of cases you simply want to run a
different program from scratch and you don't want to rely
on {\Tt{}system()} which may be dangerous and doesn't provide
much control.

If you're trying to port an application to systems where 
processes can't be forked and continue in the midst of
execution (i.e. MS-Windows or AS/400) your program's 
logic is easier to understand if you can use just one 
function call and keep the rest in a separate library.

This one-function-library gives you a {\Tt{}\nwlinkedidentq{{\_}spawnvpe}{NW1b9oex-3JXb8d-1}()} according to the 
Micro\-soft\copyright{} specification but keeping GCC (DJGPP) 
in mind. It doesn't implement the mode flag {\Tt{}P{\_}OVERLAY} 
because I can't see any use in calling spawn when I really 
mean exec.

\subsection {Synopsis}

{\tt\# include "nwprocess.h"\\[2ex]
int \_spawnvpe (\\
\hspace*{1em}int mode,\\ 
\hspace*{1em}const char *filename,\\
\hspace*{1em}char *const argv[],\\
\hspace*{1em}char *const envp[]);\\[2ex]
int \_spawnvp (\\
\hspace*{1em}int mode,\\ 
\hspace*{1em}const char *filename,\\
\hspace*{1em}char *const argv[]);}\\[2ex]
\textbf{Note:} {\Tt{}\nwlinkedidentq{{\_}spawnvp}{NW1b9oex-3JXb8d-1}()} is implemented as a macro.\\
Please also note that the parameter declaration is taken from GCC. 
Microsoft\copyright{} uses
\begin {quote}
  {\Tt{}const\ char\ *const\ *argv}
\end {quote}
which is even more weird. They both deserve an award for the
most useless use of {\Tt{}const}.

\subsection {Description}

{\Tt{}\nwlinkedidentq{{\_}spawnvpe}{NW1b9oex-3JXb8d-1}} creates and executes a process. It searches
the PATH for the new process' image. For details on this,
please have a look at your system's documentation of the
{\Tt{}execvp()} function.

\subsubsection {Function parameters}
{\Tt{}\nwlinkedidentq{{\_}spawnvpe}{NW1b9oex-3JXb8d-1}} takes the following arguments:
\begin{itemize}
  \item{{\Tt{}mode}---the mode flag may be:
    \begin {itemize}
      \item{{\Tt{}\nwlinkedidentq{P{\_}WAIT}{NW1b9oex-g0o7p-1}} and {\Tt{}\nwlinkedidentq{{\_}P{\_}WAIT}{NW1b9oex-g0o7p-1}}---wait until the child
        process has finished and return it's return value.}
      \item{{\Tt{}\nwlinkedidentq{P{\_}NOWAIT}{NW1b9oex-g0o7p-1}}, {\Tt{}\nwlinkedidentq{{\_}P{\_}NOWAIT}{NW1b9oex-g0o7p-1}} and {\Tt{}\nwlinkedidentq{{\_}P{\_}NOWAITO}{NW1b9oex-g0o7p-1}}---return
        immediately with the child's PID as the return value.}
    \end {itemize}
  }
  \item{{\Tt{}filename}---a pointer to a string containing the name of
    the child's executable image. If it is in the PATH, no path 
    information is necessary.
    
    {\Tt{}filename} must be identical to {\Tt{}argv[0]}, giving you the chance
      for an additional error.}
  \item{{\Tt{}argv}---an array of pointers to strings containing the
    command line arguments of the called process. This array shall 
    be terminated with the null-pointer constant NULL.
    
    As a very wise decision in interface design {\Tt{}argv[0]} has to be
    the same as {\Tt{}filename}.}
  \item{{\Tt{}envp}---an array of pointers to environment variables. This, too,
    has to be terminated with NULL.
    
    {\Tt{}\nwlinkedidentq{spawnvp}{NW1b9oex-3JXb8d-1}} doesn't take this argument---it just uses the default 
    environment of the caller.}
\end{itemize}

You'll find examples of calls to {\Tt{}\nwlinkedidentq{{\_}spawnvpe}{NW1b9oex-3JXb8d-1}()} on p. \pageref{test}. 

\subsection {Return value}
\subsubsection {From synchronous {\Tt{}\nwlinkedidentq{{\_}spawnvpe}{NW1b9oex-3JXb8d-1}()}}

The return value from a synchronous {\Tt{}\nwlinkedidentq{{\_}spawnvpe}{NW1b9oex-3JXb8d-1}()}
(when {\Tt{}\nwlinkedidentq{{\_}P{\_}WAIT}{NW1b9oex-g0o7p-1}} was specified for mode) is the exit 
status of the new process. Please see {\Tt{}man\ 2\ waitpid}
for details on Unix systems.\\[2ex]
\textbf{Important:} GNU-{\Tt{}\nwlinkedidentq{spawnvpe}{NW1b9oex-3JXb8d-1}} is identical to
the Unix-Versions of {\Tt{}exec} and returns the exit code of the child 
process in the lower 8 bits of the return value which
our version does not.

\subsubsection {From asynchronous {\Tt{}\nwlinkedidentq{{\_}spawnvpe}{NW1b9oex-3JXb8d-1}()}}
When {\Tt{}\nwlinkedidentq{{\_}P{\_}NOWAIT}{NW1b9oex-g0o7p-1}} or {\Tt{}\nwlinkedidentq{{\_}P{\_}NOWAITO}{NW1b9oex-g0o7p-1}} are specified for {\Tt{}mode}
the return value from {\Tt{}\nwlinkedidentq{{\_}spawnvpe}{NW1b9oex-3JXb8d-1}()} is the process id
of the child process.\\[2ex]
On error {\Tt{}\nwlinkedidentq{{\_}spawnvpe}{NW1b9oex-3JXb8d-1}()} returns -1 and {\Tt{}errno} is set. Please
refer to the man pages of your system for possible error
values. For asynchronous {\Tt{}spawn} you'll find these in the
the documentation to {\Tt{}fork()} for synchronous {\Tt{}spawn} you'll
have to look at {\Tt{}fork()} and {\Tt{}execvp()}.

\section {The main file {\Tt{}\nwlinkedidentq{spawnvpe}{NW1b9oex-3JXb8d-1}.c}}

{\Tt{}WIN32} is a standard define in Microsoft's 32 bit ML. We won't
have anything to do then.
\nwenddocs{}\nwbegincode{3}\sublabel{NW1b9oex-VqLz4-1}\nwmargintag{{\nwtagstyle{}\subpageref{NW1b9oex-VqLz4-1}}}\moddef{spawnvpe.c~{\nwtagstyle{}\subpageref{NW1b9oex-VqLz4-1}}}\endmoddef\nwstartdeflinemarkup\nwenddeflinemarkup
#ifndef WIN32
\LA{}The include files in \code{}spawnvpe.c\edoc{}~{\nwtagstyle{}\subpageref{NW1b9oex-3tsKu9-1}}\RA{}
int 
\nwlinkedidentc{_spawnvpe}{NW1b9oex-3JXb8d-1} (
  int mode, 
  const char *filename, 
  char *const argv[], 
  char *const envp[] )
\{
  extern char **environ;
  pid_t pid;
  \LA{}Assert yourself~{\nwtagstyle{}\subpageref{NW1b9oex-3iW6L4-1}}\RA{}
  pid = fork();
  if (mode == \nwlinkedidentc{P_WAIT}{NW1b9oex-g0o7p-1})
  \{
    \LA{}Exec and wait for child process~{\nwtagstyle{}\subpageref{NW1b9oex-1EndFn-1}}\RA{}
  \}
  else
  \{
    \LA{}Exec without waiting for child process~{\nwtagstyle{}\subpageref{NW1b9oex-1CNOWv-1}}\RA{}
  \}
  return -1;
\}
#endif
\nwnotused{spawnvpe.c}\nwidentuses{\\{{\nwixident{{\_}spawnvpe}}{:unspawnvpe}}\\{{\nwixident{P{\_}WAIT}}{P:unWAIT}}}\nwindexuse{\nwixident{{\_}spawnvpe}}{:unspawnvpe}{NW1b9oex-VqLz4-1}\nwindexuse{\nwixident{P{\_}WAIT}}{P:unWAIT}{NW1b9oex-VqLz4-1}\nwendcode{}\nwbegincode{4}\sublabel{NW1b9oex-3iW6L4-1}\nwmargintag{{\nwtagstyle{}\subpageref{NW1b9oex-3iW6L4-1}}}\moddef{Assert yourself~{\nwtagstyle{}\subpageref{NW1b9oex-3iW6L4-1}}}\endmoddef\nwstartdeflinemarkup\nwusesondefline{\\{NW1b9oex-VqLz4-1}}\nwenddeflinemarkup
assert (filename != NULL);
assert (argv != NULL);
assert (argv[0] != NULL);
assert (envp != NULL);
\nwused{\\{NW1b9oex-VqLz4-1}}\nwendcode{}\nwbegincode{5}\sublabel{NW1b9oex-1EndFn-1}\nwmargintag{{\nwtagstyle{}\subpageref{NW1b9oex-1EndFn-1}}}\moddef{Exec and wait for child process~{\nwtagstyle{}\subpageref{NW1b9oex-1EndFn-1}}}\endmoddef\nwstartdeflinemarkup\nwusesondefline{\\{NW1b9oex-VqLz4-1}}\nwenddeflinemarkup
int status;
if (pid == 0)
\{
  \LA{}Redirect the environment pointer~{\nwtagstyle{}\subpageref{NW1b9oex-XZV91-1}}\RA{}
  execvp (filename, argv);
\}
else
\{
  pid = waitpid (pid, &status, 0);
  if (pid == -1)
  \{
    return pid;
  \}
  if (WIFEXITED (status))
  \{
    return ((signed char)WEXITSTATUS (status));
  \}
  else
  \{
    return 0;
  \}
\}
\nwused{\\{NW1b9oex-VqLz4-1}}\nwendcode{}\nwbegincode{6}\sublabel{NW1b9oex-1CNOWv-1}\nwmargintag{{\nwtagstyle{}\subpageref{NW1b9oex-1CNOWv-1}}}\moddef{Exec without waiting for child process~{\nwtagstyle{}\subpageref{NW1b9oex-1CNOWv-1}}}\endmoddef\nwstartdeflinemarkup\nwusesondefline{\\{NW1b9oex-VqLz4-1}}\nwenddeflinemarkup
if (pid == 0)
\{
  \LA{}Redirect the environment pointer~{\nwtagstyle{}\subpageref{NW1b9oex-XZV91-1}}\RA{}
  execvp (filename, argv);
\}
else
\{
  return pid;
\}
\nwused{\\{NW1b9oex-VqLz4-1}}\nwendcode{}\nwbegindocs{7}\nwdocspar
This one is dirty for two reasons:
\begin {enumerate}
  \item {One must never ever cast away a {\Tt{}const}-specifier. On the
    other hand a {\Tt{}const} in a prototype in C isn't very much more
    than a promise not to change the caller's pointers---but
    we don't and the child can't.}
  \item {I don't know if it really works everywhere. 
    I've looked in several UNIX man pages 
    (Linux, BSD, Solaris, HPUX) but there seems to be
    no standard method to get the {\Tt{}PATH} for {\Tt{}execvp()}---it may 
    in the worst case only search `{\Tt{}/bin:/usr/bin}' after this 
    redirect of {\Tt{}environ}. So it may be wise to include the {\Tt{}PATH}
    in {\Tt{}envp} before the call to {\Tt{}\nwlinkedidentq{{\_}spawnvpe}{NW1b9oex-3JXb8d-1}()}.
    
    To get the full functionality MS-Windows\texttrademark{} has in 
    this respect, one would have to search the {\Tt{}PATH} first, store
    the pathname of the found executable in a buffer and only after
    that assign {\Tt{}environ} and call {\Tt{}execv()}.}
\end {enumerate}
\nwenddocs{}\nwbegincode{8}\sublabel{NW1b9oex-XZV91-1}\nwmargintag{{\nwtagstyle{}\subpageref{NW1b9oex-XZV91-1}}}\moddef{Redirect the environment pointer~{\nwtagstyle{}\subpageref{NW1b9oex-XZV91-1}}}\endmoddef\nwstartdeflinemarkup\nwusesondefline{\\{NW1b9oex-1EndFn-1}\\{NW1b9oex-1CNOWv-1}}\nwenddeflinemarkup
environ = (char **) envp;
\nwused{\\{NW1b9oex-1EndFn-1}\\{NW1b9oex-1CNOWv-1}}\nwendcode{}\nwbegincode{9}\sublabel{NW1b9oex-3tsKu9-1}\nwmargintag{{\nwtagstyle{}\subpageref{NW1b9oex-3tsKu9-1}}}\moddef{The include files in \code{}spawnvpe.c\edoc{}~{\nwtagstyle{}\subpageref{NW1b9oex-3tsKu9-1}}}\endmoddef\nwstartdeflinemarkup\nwusesondefline{\\{NW1b9oex-VqLz4-1}}\nwenddeflinemarkup
#include <unistd.h>
#include <sys/types.h>
#include <sys/wait.h>
#include <assert.h>
#include "nwprocess.h"
\nwused{\\{NW1b9oex-VqLz4-1}}\nwendcode{}\nwbegindocs{10}\nwdocspar
\section {The header file {\Tt{}nwprocess.h}}

This file must be included to use our Unix-version of {\Tt{}\nwlinkedidentq{{\_}spawnvpe}{NW1b9oex-3JXb8d-1}}.
On Microsoft\copyright{} Windows\texttrademark{} systems it only
includes {\Tt{}process.h}.

\nwenddocs{}\nwbegincode{11}\sublabel{NW1b9oex-4cxXo7-1}\nwmargintag{{\nwtagstyle{}\subpageref{NW1b9oex-4cxXo7-1}}}\moddef{nwprocess.h~{\nwtagstyle{}\subpageref{NW1b9oex-4cxXo7-1}}}\endmoddef\nwstartdeflinemarkup\nwenddeflinemarkup
#ifndef NWPROCESS_INCLUDED
#define NWPROCESS_INCLUDED
  #ifdef WIN32
    #include <process.h>
  #else
    \LA{}\code{}spawnvpe()\edoc{} prototype and macros~{\nwtagstyle{}\subpageref{NW1b9oex-3JXb8d-1}}\RA{}
    \LA{}\code{}spawnvpe()\edoc{} constants~{\nwtagstyle{}\subpageref{NW1b9oex-g0o7p-1}}\RA{}
  #endif
#endif
\nwnotused{nwprocess.h}\nwendcode{}\nwbegindocs{12}\nwdocspar
\subsection {Prototype and macros}
\nwenddocs{}\nwbegincode{13}\sublabel{NW1b9oex-3JXb8d-1}\nwmargintag{{\nwtagstyle{}\subpageref{NW1b9oex-3JXb8d-1}}}\moddef{\code{}spawnvpe()\edoc{} prototype and macros~{\nwtagstyle{}\subpageref{NW1b9oex-3JXb8d-1}}}\endmoddef\nwstartdeflinemarkup\nwusesondefline{\\{NW1b9oex-4cxXo7-1}}\nwenddeflinemarkup
int \nwlinkedidentc{_spawnvpe}{NW1b9oex-3JXb8d-1} (
  int mode, 
  const char *filename, 
  char * const argv[], 
  char *const envp[] );

#define \nwlinkedidentc{spawnvpe}{NW1b9oex-3JXb8d-1} \nwlinkedidentc{_spawnvpe}{NW1b9oex-3JXb8d-1}
#define \nwlinkedidentc{_spawnvp}{NW1b9oex-3JXb8d-1}(mode, filename, argvect) \\
  \nwlinkedidentc{_spawnvpe}{NW1b9oex-3JXb8d-1} ((mode), (filename), (argvect), environ);
#define \nwlinkedidentc{spawnvp}{NW1b9oex-3JXb8d-1}(mode, filename, argvect) \\
  \nwlinkedidentc{_spawnvpe}{NW1b9oex-3JXb8d-1} ((mode), (filename), (argvect), environ);
\nwindexdefn{\nwixident{{\_}spawnvpe}}{:unspawnvpe}{NW1b9oex-3JXb8d-1}\nwindexdefn{\nwixident{spawnvpe}}{spawnvpe}{NW1b9oex-3JXb8d-1}\nwindexdefn{\nwixident{{\_}spawnvp}}{:unspawnvp}{NW1b9oex-3JXb8d-1}\nwindexdefn{\nwixident{spawnvp}}{spawnvp}{NW1b9oex-3JXb8d-1}\eatline
\nwused{\\{NW1b9oex-4cxXo7-1}}\nwidentdefs{\\{{\nwixident{{\_}spawnvp}}{:unspawnvp}}\\{{\nwixident{{\_}spawnvpe}}{:unspawnvpe}}\\{{\nwixident{spawnvp}}{spawnvp}}\\{{\nwixident{spawnvpe}}{spawnvpe}}}\nwendcode{}\nwbegindocs{14}\nwdocspar
\nwenddocs{}\nwbegindocs{15}\nwdocspar
\subsection {Constants}
Please note that DJGPP defines {\Tt{}\nwlinkedidentq{P{\_}WAIT}{NW1b9oex-g0o7p-1}} as 1 and {\Tt{}\nwlinkedidentq{P{\_}NOWAIT}{NW1b9oex-g0o7p-1}} as 2. 
The following values conform to the Microsoft\copyright{} {\Tt{}process.h}.
\nwenddocs{}\nwbegincode{16}\sublabel{NW1b9oex-g0o7p-1}\nwmargintag{{\nwtagstyle{}\subpageref{NW1b9oex-g0o7p-1}}}\moddef{\code{}spawnvpe()\edoc{} constants~{\nwtagstyle{}\subpageref{NW1b9oex-g0o7p-1}}}\endmoddef\nwstartdeflinemarkup\nwusesondefline{\\{NW1b9oex-4cxXo7-1}}\nwenddeflinemarkup
#ifndef \nwlinkedidentc{_P_WAIT}{NW1b9oex-g0o7p-1}
  #define \nwlinkedidentc{_P_WAIT}{NW1b9oex-g0o7p-1} 0
#endif
#ifndef \nwlinkedidentc{P_WAIT}{NW1b9oex-g0o7p-1}
  #define \nwlinkedidentc{P_WAIT}{NW1b9oex-g0o7p-1} \nwlinkedidentc{_P_WAIT}{NW1b9oex-g0o7p-1}
#endif

#ifndef \nwlinkedidentc{_P_NOWAIT}{NW1b9oex-g0o7p-1}
  #define \nwlinkedidentc{_P_NOWAIT}{NW1b9oex-g0o7p-1} 1
#endif
#ifndef \nwlinkedidentc{P_NOWAIT}{NW1b9oex-g0o7p-1}
  #define \nwlinkedidentc{P_NOWAIT}{NW1b9oex-g0o7p-1} \nwlinkedidentc{_P_NOWAIT}{NW1b9oex-g0o7p-1}
#endif

#ifndef \nwlinkedidentc{_P_NOWAITO}{NW1b9oex-g0o7p-1}
  #define \nwlinkedidentc{_P_NOWAITO}{NW1b9oex-g0o7p-1} 3
#endif
\nwindexdefn{\nwixident{{\_}P{\_}WAIT}}{:unP:unWAIT}{NW1b9oex-g0o7p-1}\nwindexdefn{\nwixident{P{\_}WAIT}}{P:unWAIT}{NW1b9oex-g0o7p-1}\nwindexdefn{\nwixident{{\_}P{\_}NOWAIT}}{:unP:unNOWAIT}{NW1b9oex-g0o7p-1}\nwindexdefn{\nwixident{P{\_}NOWAIT}}{P:unNOWAIT}{NW1b9oex-g0o7p-1}\nwindexdefn{\nwixident{{\_}P{\_}NOWAITO}}{:unP:unNOWAITO}{NW1b9oex-g0o7p-1}\eatline
\nwused{\\{NW1b9oex-4cxXo7-1}}\nwidentdefs{\\{{\nwixident{{\_}P{\_}NOWAIT}}{:unP:unNOWAIT}}\\{{\nwixident{{\_}P{\_}NOWAITO}}{:unP:unNOWAITO}}\\{{\nwixident{{\_}P{\_}WAIT}}{:unP:unWAIT}}\\{{\nwixident{P{\_}NOWAIT}}{P:unNOWAIT}}\\{{\nwixident{P{\_}WAIT}}{P:unWAIT}}}\nwendcode{}\nwbegindocs{17}\nwdocspar
% spawntest.nw
% June 2000 by Kai Uwe Karl Baumbach, baumbach@kabece.com
\nwenddocs{}\nwbegindocs{18}\nwdocspar
\section {A small test suite} \label{test}

This `test suite' requires GNU GCC. Basically one should write
a test routine for every piece of code. This one may also serve as 
an example for the usage of {\Tt{}\nwlinkedidentq{spawnvpe}{NW1b9oex-3JXb8d-1}()}.

\subsection {{\Tt{}spawntest.exe}}
\nwenddocs{}\nwbegincode{19}\sublabel{NW1b9oex-1u8iTI-1}\nwmargintag{{\nwtagstyle{}\subpageref{NW1b9oex-1u8iTI-1}}}\moddef{spawntest.c~{\nwtagstyle{}\subpageref{NW1b9oex-1u8iTI-1}}}\endmoddef\nwstartdeflinemarkup\nwenddeflinemarkup
#include <stdlib.h> 
#include <stdio.h> 
#include <errno.h>
#include <string.h>
#include "nwprocess.h"

int 
\nwlinkedidentc{main}{NW1b9oex-1u8iTI-1} (void)
\{
  char *environment[] = \{
    "THIS=is",
    "THE=environment",
    "TO=be",
    "PASSED=to",
    "THE=child",
    "PROCESS=end",
    NULL
  \};
  char *argvector[] = \{
    "spawnchild.exe",
    "First Parameter",
    "Second Parameter",
    "Third Parameter",
    NULL
  \};
  int retval;
  int wait_mode;
  char *program_name;
  
  program_name = "spawnchild.exe";

  wait_mode = \nwlinkedidentc{P_NOWAIT}{NW1b9oex-g0o7p-1};
  \LA{}Spawn a child process and display the results~{\nwtagstyle{}\subpageref{NW1b9oex-1h2skZ-1}}\RA{}
  wait_mode = \nwlinkedidentc{P_WAIT}{NW1b9oex-g0o7p-1};
  \LA{}Spawn a child process and display the results~{\nwtagstyle{}\subpageref{NW1b9oex-1h2skZ-1}}\RA{} 

  program_name = "ls";
  argvector[0] = program_name;
  argvector[1] = "-Flias";
  argvector[2] = NULL;
  
  wait_mode = \nwlinkedidentc{_P_NOWAITO}{NW1b9oex-g0o7p-1};
  \LA{}Spawn a child process and display the results~{\nwtagstyle{}\subpageref{NW1b9oex-1h2skZ-1}}\RA{}
  wait_mode = \nwlinkedidentc{P_WAIT}{NW1b9oex-g0o7p-1};
  \LA{}Spawn a child process and display the results~{\nwtagstyle{}\subpageref{NW1b9oex-1h2skZ-1}}\RA{} 

  program_name = "program_with_a_silly_name";
  argvector[0] = program_name;
  argvector[1] = NULL;
  
  wait_mode = \nwlinkedidentc{_P_WAIT}{NW1b9oex-g0o7p-1};
  \LA{}Spawn a child process and display the results~{\nwtagstyle{}\subpageref{NW1b9oex-1h2skZ-1}}\RA{} 
  
  program_name = "sh";
    argvector[0] = program_name;
    argvector[1] = "-c";
    argvector[2] = "set";
    argvector[3] = NULL;

    wait_mode = \nwlinkedidentc{_P_WAIT}{NW1b9oex-g0o7p-1};
    \LA{}Spawn a child process and display the results~{\nwtagstyle{}\subpageref{NW1b9oex-1h2skZ-1}}\RA{}
  
  return EXIT_SUCCESS;
\}
\nwindexdefn{\nwixident{main}}{main}{NW1b9oex-1u8iTI-1}\eatline
\nwnotused{spawntest.c}\nwidentdefs{\\{{\nwixident{main}}{main}}}\nwidentuses{\\{{\nwixident{{\_}P{\_}NOWAITO}}{:unP:unNOWAITO}}\\{{\nwixident{{\_}P{\_}WAIT}}{:unP:unWAIT}}\\{{\nwixident{P{\_}NOWAIT}}{P:unNOWAIT}}\\{{\nwixident{P{\_}WAIT}}{P:unWAIT}}}\nwindexuse{\nwixident{{\_}P{\_}NOWAITO}}{:unP:unNOWAITO}{NW1b9oex-1u8iTI-1}\nwindexuse{\nwixident{{\_}P{\_}WAIT}}{:unP:unWAIT}{NW1b9oex-1u8iTI-1}\nwindexuse{\nwixident{P{\_}NOWAIT}}{P:unNOWAIT}{NW1b9oex-1u8iTI-1}\nwindexuse{\nwixident{P{\_}WAIT}}{P:unWAIT}{NW1b9oex-1u8iTI-1}\nwendcode{}\nwbegincode{20}\sublabel{NW1b9oex-1h2skZ-1}\nwmargintag{{\nwtagstyle{}\subpageref{NW1b9oex-1h2skZ-1}}}\moddef{Spawn a child process and display the results~{\nwtagstyle{}\subpageref{NW1b9oex-1h2skZ-1}}}\endmoddef\nwstartdeflinemarkup\nwusesondefline{\\{NW1b9oex-1u8iTI-1}}\nwenddeflinemarkup
printf (
  "Spawning %s in mode _P_%sWAIT:\\n",
  program_name,
  wait_mode == \nwlinkedidentc{P_WAIT}{NW1b9oex-g0o7p-1} ? "" : "NO" );
retval = \nwlinkedidentc{_spawnvpe}{NW1b9oex-3JXb8d-1} (
  wait_mode,
  program_name,
  argvector,
  environment);
printf ("The return value was %d.\\n",
  retval);
if (retval == -1)
\{
  printf (
    "errno = %d: %s.\\n", 
    errno,
    strerror (errno));
\}
\nwused{\\{NW1b9oex-1u8iTI-1}}\nwidentuses{\\{{\nwixident{{\_}spawnvpe}}{:unspawnvpe}}\\{{\nwixident{P{\_}WAIT}}{P:unWAIT}}}\nwindexuse{\nwixident{{\_}spawnvpe}}{:unspawnvpe}{NW1b9oex-1h2skZ-1}\nwindexuse{\nwixident{P{\_}WAIT}}{P:unWAIT}{NW1b9oex-1h2skZ-1}\nwendcode{}\nwbegindocs{21}\nwdocspar
\subsection {{\Tt{}spawnchild.exe}}

Note that the definition of {\Tt{}\nwlinkedidentq{main}{NW1b9oex-1u8iTI-1}()} in this program is not ISO-C
and is deprecated even on systems where it is available. It nevertheless
is a nice test to see, if {\Tt{}environ\ =\ envp} before the call to {\Tt{}execvp()}
really works.
\nwenddocs{}\nwbegincode{22}\sublabel{NW1b9oex-1nyD9J-1}\nwmargintag{{\nwtagstyle{}\subpageref{NW1b9oex-1nyD9J-1}}}\moddef{spawnchild.c~{\nwtagstyle{}\subpageref{NW1b9oex-1nyD9J-1}}}\endmoddef\nwstartdeflinemarkup\nwenddeflinemarkup
#include <stdlib.h> 
#include <stdio.h> 
#ifndef WIN32
  #include <unistd.h> 
#else
  #define sleep(x) (void)0
#endif

int 
\nwlinkedidentc{main}{NW1b9oex-1u8iTI-1} (int argc, char *argv[], char *envp[])
\{
  sleep (1);
  puts ("Environment in spawnchild.exe:");
  \LA{}Print the process environment~{\nwtagstyle{}\subpageref{NW1b9oex-2t5zF7-1}}\RA{}
  puts ("\\nParameters in spawnchild.exe:");
  \LA{}Print the process parameters~{\nwtagstyle{}\subpageref{NW1b9oex-4EVmeQ-1}}\RA{}
  puts ("\\nVariable \\"THIS\\" in spwanchild:");
  printf ("%s\\n", getenv ("THIS"));
  return EXIT_SUCCESS;
\}
\nwindexdefn{\nwixident{main}}{main}{NW1b9oex-1nyD9J-1}\eatline
\nwnotused{spawnchild.c}\nwidentdefs{\\{{\nwixident{main}}{main}}}\nwendcode{}\nwbegincode{23}\sublabel{NW1b9oex-4EVmeQ-1}\nwmargintag{{\nwtagstyle{}\subpageref{NW1b9oex-4EVmeQ-1}}}\moddef{Print the process parameters~{\nwtagstyle{}\subpageref{NW1b9oex-4EVmeQ-1}}}\endmoddef\nwstartdeflinemarkup\nwusesondefline{\\{NW1b9oex-1nyD9J-1}}\nwenddeflinemarkup
while (*argv)
\{
  printf ( "%s\\n", *argv);
  argv++;
\}
\nwused{\\{NW1b9oex-1nyD9J-1}}\nwendcode{}\nwbegincode{24}\sublabel{NW1b9oex-2t5zF7-1}\nwmargintag{{\nwtagstyle{}\subpageref{NW1b9oex-2t5zF7-1}}}\moddef{Print the process environment~{\nwtagstyle{}\subpageref{NW1b9oex-2t5zF7-1}}}\endmoddef\nwstartdeflinemarkup\nwusesondefline{\\{NW1b9oex-1nyD9J-1}}\nwenddeflinemarkup
while (*envp)
\{
  printf ( "%s\\n", *envp);
  envp++;
\}
\nwused{\\{NW1b9oex-1nyD9J-1}}\nwendcode{}\nwbegincode{25}\sublabel{NW1b9oex-4QIgwf-1}\nwmargintag{{\nwtagstyle{}\subpageref{NW1b9oex-4QIgwf-1}}}\moddef{spawntest.mak~{\nwtagstyle{}\subpageref{NW1b9oex-4QIgwf-1}}}\endmoddef\nwstartdeflinemarkup\nwenddeflinemarkup
CC=gcc -Wall -pedantic

all:
  $\{CC\} -o spawntest.exe \nwlinkedidentc{spawnvpe}{NW1b9oex-3JXb8d-1}.c spawntest.c
  $\{CC\} -o spawnchild.exe spawnchild.c
  ./spawntest.exe


\nwnotused{spawntest.mak}\nwidentuses{\\{{\nwixident{spawnvpe}}{spawnvpe}}}\nwindexuse{\nwixident{spawnvpe}}{spawnvpe}{NW1b9oex-4QIgwf-1}\nwendcode{}

\nwixlogsorted{c}{{\code{}spawnvpe()\edoc{} constants}{NW1b9oex-g0o7p-1}{\nwixu{NW1b9oex-4cxXo7-1}\nwixd{NW1b9oex-g0o7p-1}}}%
\nwixlogsorted{c}{{\code{}spawnvpe()\edoc{} prototype and macros}{NW1b9oex-3JXb8d-1}{\nwixu{NW1b9oex-4cxXo7-1}\nwixd{NW1b9oex-3JXb8d-1}}}%
\nwixlogsorted{c}{{Assert yourself}{NW1b9oex-3iW6L4-1}{\nwixu{NW1b9oex-VqLz4-1}\nwixd{NW1b9oex-3iW6L4-1}}}%
\nwixlogsorted{c}{{Exec and wait for child process}{NW1b9oex-1EndFn-1}{\nwixu{NW1b9oex-VqLz4-1}\nwixd{NW1b9oex-1EndFn-1}}}%
\nwixlogsorted{c}{{Exec without waiting for child process}{NW1b9oex-1CNOWv-1}{\nwixu{NW1b9oex-VqLz4-1}\nwixd{NW1b9oex-1CNOWv-1}}}%
\nwixlogsorted{c}{{nwprocess.h}{NW1b9oex-4cxXo7-1}{\nwixd{NW1b9oex-4cxXo7-1}}}%
\nwixlogsorted{c}{{Print the process environment}{NW1b9oex-2t5zF7-1}{\nwixu{NW1b9oex-1nyD9J-1}\nwixd{NW1b9oex-2t5zF7-1}}}%
\nwixlogsorted{c}{{Print the process parameters}{NW1b9oex-4EVmeQ-1}{\nwixu{NW1b9oex-1nyD9J-1}\nwixd{NW1b9oex-4EVmeQ-1}}}%
\nwixlogsorted{c}{{Redirect the environment pointer}{NW1b9oex-XZV91-1}{\nwixu{NW1b9oex-1EndFn-1}\nwixu{NW1b9oex-1CNOWv-1}\nwixd{NW1b9oex-XZV91-1}}}%
\nwixlogsorted{c}{{Spawn a child process and display the results}{NW1b9oex-1h2skZ-1}{\nwixu{NW1b9oex-1u8iTI-1}\nwixd{NW1b9oex-1h2skZ-1}}}%
\nwixlogsorted{c}{{spawnchild.c}{NW1b9oex-1nyD9J-1}{\nwixd{NW1b9oex-1nyD9J-1}}}%
\nwixlogsorted{c}{{spawntest.c}{NW1b9oex-1u8iTI-1}{\nwixd{NW1b9oex-1u8iTI-1}}}%
\nwixlogsorted{c}{{spawntest.mak}{NW1b9oex-4QIgwf-1}{\nwixd{NW1b9oex-4QIgwf-1}}}%
\nwixlogsorted{c}{{spawnvpe.c}{NW1b9oex-VqLz4-1}{\nwixd{NW1b9oex-VqLz4-1}}}%
\nwixlogsorted{c}{{The include files in \code{}spawnvpe.c\edoc{}}{NW1b9oex-3tsKu9-1}{\nwixu{NW1b9oex-VqLz4-1}\nwixd{NW1b9oex-3tsKu9-1}}}%
\nwixlogsorted{i}{{\nwixident{{\_}P{\_}NOWAIT}}{:unP:unNOWAIT}}%
\nwixlogsorted{i}{{\nwixident{{\_}P{\_}NOWAITO}}{:unP:unNOWAITO}}%
\nwixlogsorted{i}{{\nwixident{{\_}P{\_}WAIT}}{:unP:unWAIT}}%
\nwixlogsorted{i}{{\nwixident{{\_}spawnvp}}{:unspawnvp}}%
\nwixlogsorted{i}{{\nwixident{{\_}spawnvpe}}{:unspawnvpe}}%
\nwixlogsorted{i}{{\nwixident{main}}{main}}%
\nwixlogsorted{i}{{\nwixident{P{\_}NOWAIT}}{P:unNOWAIT}}%
\nwixlogsorted{i}{{\nwixident{P{\_}WAIT}}{P:unWAIT}}%
\nwixlogsorted{i}{{\nwixident{spawnvp}}{spawnvp}}%
\nwixlogsorted{i}{{\nwixident{spawnvpe}}{spawnvpe}}%
\nwbegindocs{26}\nwdocspar
\section {Index}
\subsection {Chunks}
\nowebchunks{}
\subsection{Identifiers}
\nowebindex{}
\end{document}
\nwenddocs{}
